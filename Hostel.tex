
\documentclass{article}

\setlength{\oddsidemargin}{0.25in}

\setlength{\textwidth}{7in}

\setlength{\topmargin}{-0.25in}

\setlength{\textheight}{9in}

\begin{document}

\title{STUDY INVESTIGATING THE IMPACT OF HOSTEL LIFE.}         
\author{KAGGWA HAM 14/U/6759/PS   214018321}        
\date{17/05/2017}          
\maketitle

\section{Introduction}       
\paragraph{The study explored the impacts of hostel life on the behavior, and personality of the students. Sample consisted of five hostel students together with their hostels of resident both male and female hostel students and the age range was twenty to twenty-five. ODK COLLECT was used to conceptualize the findings. Results revealed that hostels have great importance in the educational journey of students. Hostel life expands the social circle of the hostel students, because hostel is a combination of multicultural social group. The personality characteristics associated with the hostel students are such as they are considered to be confident, punctual, social, realistic, compromising, responsible, and sharp in many domains of life. During hostel stay, students learn to live with different types of individuals, and hostel life also increases the students� level of patience. It prepares students to accept challenges in practical life. Individual differences are very common among the hostel roommates. Majority of the male hostel students are affected negatively due to drug use.
Result of the study can help to improve quality of hostel services which may increase student�s hostel life satisfaction.}
\section{Background }
\paragraph{Education has always played an important role in student life and it has been conducted in various ways depending on the culture and location. Human personality is shaped by the experiences of life. When a child is born the family provides a protective environment for the child, at the beginning the interactions are limited latter social interactions increase, and the process of socialization starts.
This enables the individuals to become an effective member of a society, Human�s lifestyle and personality is affected by his/her surroundings. Therefore the social structure plays a vital role in the development of personality and behavior.
Researches have been conducted to highlight the importance of the home environment, and the role of family members in the development of the children. Differences in child development start with the social economic status of the family, biological endowment and educational differences. These family differences make enduring changes in the personality of the children. As family members play an important role in the development of children. Therefore a cooperative family environment inculcates confidence in children. Residential areas also affect the process of socialization. This in turn limits or expands educational opportunities for children. 
Education is a part of child develop, it started with the birth and lasted till the time of death. It is a process in which an individual learns new skills and information. The main goal of education is to encourage the individual to acquire tasks, knowledge, facts, and traits which previously not obtained. Human society depended to learn, where parents and other members from our society can facilitate the process of learning. [1] Callaway, 1979.
Since Uganda is an underdeveloped country, very limited budget is reserved for the education department, higher education facilities are only accessible in developing and big parts of the country. Because of that students need to stay in hostel for higher education. 
Uganda  hostels play an important role in the educational journey of students. Hostels are proving residential opportunities for the students to continue the process of education.
In some countries the word hostel is specifically used for the accommodation of student and travelers. However in Uganda, the hostel is believed to be a place of residence that a school, colleges or universities has, all hostels are supervised by the hostel wardens and other staff. The hostel generally consists hundreds of students. All of them make a group of students. These students come from different ethical, social, geographical and economical background.
The hostel is a place where students stay for pursuing formal education away from their homes. But the concept of hostel is not only limited to place of residence, hostel is a human practical laboratory. Therefore hostel is not simply a place for living it is a center of education. Students learn as much as from their lecturers as well as fellows during hostel stay. It enriches the understanding of the curriculum through analytical discussion among the students living in the hostels, and may contribute to character building as well. Students in hostel not only learn the theoretical material they also learn how to enhance their personal abilities and learn to live independently. [2] Mishra, 1994.}

\section{Problem Statement}
\paragraph{Students that go in hostels  face problem of adjusting with the hostel environment and roommates. Students always said  that in starting days they missed their home and family too much and sometime after talking with parents on cell phones they felt like going back home due to the hostel life style they did not know more in their carrier.}
\section{Objectives}
\subsection{Main Objective}
\paragraph{To investigate the impact of hostel life on students.}
\subsubsection{Specific Objectives}
\paragraph{To explore the perception of students about hostel life.}
\paragraph{To study gender differences among hostel students.}
\paragraph{To study the psychological and behavioral impacts of hostel life.}
\paragraph{To study the influence of the multicultural peer group.}
\paragraph{To make suitable recommendation for the improvement of hostel facilities.}
\section{Scope}
\paragraph{This investigation focused on internal and external students of Makerere University living around makerere kikoni.}
\section{Significance}
\paragraph{The study investigated the perceptions of students about their hostel life as well as the impacts of hostel life on students. The study helps the teachers and parents of students to understand the importance of hostel life, and academic performance, moreover the study helps students to overcome problems they face during hostel stay. Results of the study will help to improve hostel life of students.}
\section{Methodology}
\subsection{Participants}
\paragraph{A sample of five hostel students was selected from different hostels around makerere kikoni and convenience sampling was used to select the hostels.
 The sample consisted of both male and female hostel students. For attaining maximum variation in sample, different students were selected from different parts of makerere kikoni like Sunways hostel Baskon hostel Kann hostel Muhika hostel and Olympia hostel.
The reason for selecting these students was that they represent the cultures of their hostels and they provide better information about their experience of hostel life.} 
\subsection{Material}
\paragraph{ODK COLLECT  was used for data collection. This included asking students of their hostel life style and pictures of their hostels were taken.  Also face-to-face conversations were done for gathering relevant information from hostel students.}
\section{Conclusion}
\paragraph{The study explored the impact of hostel life on hostel students. It highlighted the experiences, behavioral changes, and personality characteristics of the hostel students. It also studies the gender differences among the roommates. Results show that male hostel students are more prone to affect negatively during hostel stay as most of them indulge in drug addiction. Female hostel students adjust in hostel more easily than male students. Positive behavioral changes involved character building and preparing students for future practical life. The negative behavioral adaptations included students become lazy, show careless attitude towards  studies, wasting time with friends, smoking and drug addiction in male students.
Personality characteristics related with hostel students are as such they considered to be realistic, punctual, disciplined, independent, compromising, and well organized.}








\section{References}

\paragraph{[1] Callaway, R. (1979) Teachers� Beliefs Concerning Values and the Functions and Purposes of Schooling, Eric Document Reproduction Service No. ED 177110.}
\paragraph{[2] Mishra, A.N. (1994). Students and Hostel Life. New Delhi: Mittal publications.}

\end{document}
